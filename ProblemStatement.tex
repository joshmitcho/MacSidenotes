\documentclass{article}

\usepackage{graphicx}
\usepackage[utf8]{inputenc}
\usepackage[english]{babel}

\title{%
	SE 3XA3: Problem Statement \\
	\large MacSidenotes \vspace{-5ex}\\}
\date{}


\setlength{\parindent}{4em}
\setlength{\parskip}{1em}

\begin{document}
	\begin{table}[ht]
		\centering
		\resizebox{\textwidth}{!}{\begin{tabular}{| c c c |}
			\hline
			
			\multicolumn{3}{ | c | }{\textbf{Table 1: Revision History}} \\ 
			
			\hline
			
			\multicolumn{1}{ | l | }{\textbf{Date}} & \multicolumn{1}{   l | }{\textbf{Developer}} & 
			\multicolumn{1}{   l | }{\textbf{Change}} \\
			
			\hline
			
			\multicolumn{1}{ | l | }{Sept 21st, 2016} & \multicolumn{1}{   l | }{Josh, Matt} & 
			\multicolumn{1}{   l | }{Initial Draft} \\
			
			\hline
			
		\end{tabular}}
	\end{table}
	\newpage
	
	\maketitle
	
	\begin{center}
		
		\begin{tabular}{| c  c |}
		
			
			\multicolumn{2}{  c }{Group 4} \\ 
			
		
			
			\multicolumn{1}{ l }{Josh Mitchell} & \multicolumn{1}{ r }{mitchjp3}  \\
			
		
			
		    \multicolumn{1}{ l }{Matthew Shortt} & \multicolumn{1}{ r }{shorttmk}  \\
		
			
		
			
		\end{tabular}
	\end{center}
	
	\begin{flushleft}
		
		When surfing the web, often one will find a webpage they wish to return to. 
		Sometimes the reason to return to this page is more complicated than the ~30 
		characters that Chrome allows for bookmark descriptions before truncating them 
		with ellipses. Additionally, one may wish to annotate the pages they visit, to 
		remember past thought-processes if a task needs to be postponed and resumed 
		later. If one can’t accurately describe the reason to bookmark a page, or forgets key 
		information about that page’s relevancy, they may end up with a slew of bookmarked 
		pages they can’t remember why they saved in the first place.\par
		
		We organize our lives via technology; many of us by our traversal of the web. Online 
		disorganization can lead to forgotten tasks and missed deadlines. Spending a small 
		amount of time to create an organized and well thought out trail of digital 
		breadcrumbs prevents future headache and potential neglected commitments.\par
		
		Stakeholders for this project include anyone who is looking for a more in depth and 
		organized way of tracking where and why they were searching the web. The 
		extension would be best utilized by those who take a more thorough approach to 
		their online organization, such as students and working professionals.\par
		
		MacSidenotes is a Chrome extension built using JavaScript that one could install on 
		any operating system that is Chrome compatible, making it accessible on most 
		personal computers. Almost every consumer laptop or desktop running Windows, 
		Mac OS, Linux or Chrome OS will be able to use MacSidenotes.
		
		
	\end{flushleft}
		
	
	
\end{document}