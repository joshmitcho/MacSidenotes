\documentclass[12pt, titlepage]{article}

\usepackage{fullpage}
\usepackage[round]{natbib}
\usepackage{multirow}
\usepackage{booktabs}
\usepackage{tabularx}
\usepackage{graphicx}
\usepackage{float}
\usepackage{hyperref}
\usepackage{color}
\usepackage[normalem]{ulem}
\hypersetup{
    colorlinks,
    citecolor=black,
    filecolor=black,
    linkcolor=red,
    urlcolor=blue
}
\usepackage[round]{natbib}

\newcounter{acnum}
\newcommand{\actheacnum}{AC\theacnum}
\newcommand{\acref}[1]{AC\ref{#1}}

\newcounter{ucnum}
\newcommand{\uctheucnum}{UC\theucnum}
\newcommand{\uref}[1]{UC\ref{#1}}

\newcounter{mnum}
\newcommand{\mthemnum}{M\themnum}
\newcommand{\mref}[1]{M\ref{#1}}

\title{SE 3XA3: Module Guide\\MacSidenotes}

\author{Team 4
	\\ Josh Mitchell mitchjp3
	\\ Matthew Shortt shorttmk
}

\date{\today}

%\input{../../Comments}
%\sout{text}

\begin{document}

\maketitle

\pagenumbering{roman}
\tableofcontents
\listoftables
\listoffigures

\begin{table}[H]
\caption{\bf Revision History}
\begin{tabularx}{\textwidth}{p{3cm}p{2cm}X}
\toprule {\bf Date} & {\bf Version} & {\bf Notes}\\
\midrule
Nov 13th & Rev0 & Complete Sections 1-7\\
\textcolor{blue}{Dec 7th} & \textcolor{blue}{Rev 1.0} & \textcolor{blue}{Revision 1 
Modifications}\\
\bottomrule
\end{tabularx}
\end{table}

\newpage

\pagenumbering{arabic}

\section{Introduction}

\textcolor{blue}{MacSidenotes is a Chrome Extension for taking notes on any web site. 
Your notes are linked with the URL and will appear upon your next visit. Works offline 
using Chrome's local storage.}

\textcolor{blue}{MacSidenotes uses an MVC architecture. The model of the application 
would be the HTML, which is the skeleton for the content. The CSS of the extension 
would represent the view, as it adds the visual styling to the application. This perfectly 
embodies the separation of concerns as any changes to the CSS will not affect the 
functionality of the application. Finally, the controller would be the JS (JavaScript) file in 
tandem with the browser, in our case Google Chrome, , as it is responsible for the 
combination of the HTML and CSS as well as rendering them.  }

Decomposing a system into modules is a commonly accepted approach to developing
software.  A module is a work assignment for a programmer or programming
team~\citep{ParnasEtAl1984}.  We advocate a decomposition
based on the principle of information hiding~\citep{Parnas1972a}.  This
principle supports design for change, because the ``secrets'' that each module
hides represent likely future changes.  Design for change is valuable in SC,
where modifications are frequent, especially during initial development as the
solution space is explored.  

Our design follows the rules layed out by \citet{ParnasEtAl1984}, as follows:
\begin{itemize}
\item System details that are likely to change independently should be the
  secrets of separate modules.
\item Each data structure is used in only one module.
\item Any other program that requires information stored in a module's data
  structures must obtain it by calling access programs belonging to that module.
\end{itemize}

After completing the first stage of the design, the Software Requirements
Specification (SRS), the Module Guide (MG) is developed~\citep{ParnasEtAl1984}. The MG
specifies the modular structure of the system and is intended to allow both
designers and maintainers to easily identify the parts of the software.  The
potential readers of this document are as follows:

\begin{itemize}
\item New project members: This document can be a guide for a new project member
  to easily understand the overall structure and quickly find the
  relevant modules they are searching for.
\item Maintainers: The hierarchical structure of the module guide improves the
  maintainers' understanding when they need to make changes to the system. It is
  important for a maintainer to update the relevant sections of the document
  after changes have been made.
\item Designers: Once the module guide has been written, it can be used to
  check for consistency, feasibility and flexibility. Designers can verify the
  system in various ways, such as consistency among modules, feasibility of the
  decomposition, and flexibility of the design.
\end{itemize}

The rest of the document is organized as follows. Section
\ref{SecChange} lists the anticipated and unlikely changes of the software
requirements. Section \ref{SecMH} summarizes the module decomposition that
was constructed according to the likely changes. Section \ref{SecConnection}
specifies the connections between the software requirements and the
modules. Section \ref{SecMD} gives a detailed description of the
modules. Section \ref{SecTM} includes two traceability matrices. One checks
the completeness of the design against the requirements provided in the SRS. The
other shows the relation between anticipated changes and the modules. Section
\ref{SecUse} describes the use relation between modules.

\section{Anticipated and Unlikely Changes} \label{SecChange}

This section lists possible changes to the system. According to the likeliness
of the change, the possible changes are classified into two
categories. Anticipated changes are listed in Section \ref{SecAchange}, and
unlikely changes are listed in Section \ref{SecUchange}.

\subsection{Anticipated Changes} \label{SecAchange}

Anticipated changes are the source of the information that is to be hidden
inside the modules. Ideally, changing one of the anticipated changes will only
require changing the one module that hides the associated decision. The approach
adapted here is called design for
change.

\begin{description}
\item[\refstepcounter{acnum} \actheacnum \label{acTrigger}:] The trigger of 
the save functionality.
\item[\refstepcounter{acnum} \actheacnum \label{acUI}:] Appearance of the UI 
(eg. buttons).
\item[\refstepcounter{acnum} \actheacnum \label{acSave}:] Location \& format of 
saved notes.
\item[\refstepcounter{acnum} \actheacnum \label{acVisibility}:] Method of 
determining Master List visibility.
\item[\refstepcounter{acnum} \actheacnum \label{acUpdate}:] Method by which the 
Master List is updated.
\item[\refstepcounter{acnum} \actheacnum \label{acURL}:] Method by which the 
page URL is fetched.
\item[\refstepcounter{acnum} \actheacnum \label{acNotify}:] Method by which the 
User is notified of successful Note saves/deletions.
\end{description}

\subsection{Unlikely Changes} \label{SecUchange}

The module design should be as general as possible. However, a general system is
more complex. Sometimes this complexity is not necessary. Fixing some design
decisions at the system architecture stage can simplify the software design. If
these decision should later need to be changed, then many parts of the design
will potentially need to be modified. Hence, it is not intended that these
decisions will be changed.

\begin{description}
\item[\refstepcounter{ucnum} \uctheucnum \label{ucIO}:] Input/Output devices
  (Input: File and/or Keyboard, Output: File, Memory, and/or Screen).
\item[\refstepcounter{ucnum} \uctheucnum \label{ucInput}:] There will always be
  a source of input data external to the software.
\item[\refstepcounter{ucnum} \uctheucnum \label{ucChrome}:] The software will 
  be run within a Chrome Extension environment.
  
\item[\refstepcounter{ucnum} \uctheucnum \label{ucGoal}:] \sout{The goal of the 
	software: to save notes for the user for later viewing. }
\end{description}

\section{Module Hierarchy} \label{SecMH}

This section provides an overview of the module design. Modules are summarized
in a hierarchy decomposed by secrets in Table \ref{TblMH}. The modules listed
below, which are leaves in the hierarchy tree, are the modules that will
actually be implemented.

\begin{description}
\item [\refstepcounter{mnum} \mthemnum \label{mHH}:] Hardware-Hiding Module
\item [\refstepcounter{mnum} \mthemnum \label{mSave}:] Save Note Module
\item [\refstepcounter{mnum} \mthemnum \label{mVisibility}:] Master List 
Visibility Module
\item [\refstepcounter{mnum} \mthemnum \label{mUpdate}:] Update Master List 
Module
\item [\refstepcounter{mnum} \mthemnum \label{mURL}:] URL Fetch Module
\item [\refstepcounter{mnum} \mthemnum \label{mTrigger}:] Save Trigger Module
\item [\refstepcounter{mnum} \mthemnum \label{mNotify}:] Notify User Module
\item [\refstepcounter{mnum} \mthemnum \label{mUI}:] UI Appearance Module
\end{description}

\textbf{M1} is handled by the Chrome Extension environment on which the 
software is running. Once Chrome has been installed, the hardware is hidden 
from the rest of the software.\\

\begin{table}[h!]
\centering
\begin{tabular}{p{0.3\textwidth} p{0.6\textwidth}}
\toprule
\textbf{Level 1} & \textbf{Level 2}\\
\midrule

{Hardware-Hiding Module} & ~ \\
\midrule

\multirow{3}{0.3\textwidth}{Behaviour-Hiding Module} & Save Trigger Module\\
& Notify User Module\\
& UI Appearance Module\\
\midrule

\multirow{3}{0.3\textwidth}{Software Decision Module} & {
	Update Master List Module}\\
& Save Note Module\\
& URL Fetch Module\\
& Master List Visibility Module\\
\bottomrule

\end{tabular}
\caption{Module Hierarchy}
\label{TblMH}
\end{table}

\section{Connection Between Requirements and Design} \label{SecConnection}

The design of the system is intended to satisfy the requirements developed in
the SRS. In this stage, the system is decomposed into modules. The connection
between requirements and modules is listed in Table \ref{TblRT}.

\section{Module Decomposition} \label{SecMD}

Modules are decomposed according to the principle of ``information hiding''
proposed by \citet{ParnasEtAl1984}. The \emph{Secrets} field in a module
decomposition is a brief statement of the design decision hidden by the
module. The \emph{Services} field specifies \emph{what} the module will do
without documenting \emph{how} to do it. For each module, a suggestion for the
implementing software is given under the \emph{Implemented By} title. If the
entry is \emph{OS}, this means that the module is provided by the operating
system or by standard programming language libraries.  Also indicate if the
module will be implemented specifically for the software.

Only the leaf modules in the
hierarchy have to be implemented. If a dash (\emph{--}) is shown, this means
that the module is not a leaf and will not have to be implemented. Whether or
not this module is implemented depends on the programming language
selected.

\subsection{Hardware Hiding Modules (\mref{mHH})}

\begin{description}
\item[Secrets:]The data structure and algorithm used to implement the virtual
  hardware.
\item[Services:]Serves as a virtual hardware used by the rest of the
  system. This module provides the interface between the hardware and the
  software. So, the system can use it to display outputs or to accept inputs.
\item[Implemented By:] OS, Chrome
\end{description}

\subsection{Behaviour-Hiding Module}

\subsubsection{Save Trigger Module (\mref{mTrigger})}

\begin{description}
\item[Secrets:]What causes the Note to be saved (button press vs closing the 
sidebar).
\item[Services:]Allows the call of the Save Note module \textbf{\mref{mSave}} 
(specified below).
\item[Implemented By:] MacSidenotes
\end{description}

\subsubsection{Notify User Module (\mref{mNotify})}

\begin{description}
\item[Secrets:]The format of the user notification message.
\item[Services:]Allows the user to be alerted when they have successfully saved 
or deleted a Note.
\item[Implemented By:] MacSidenotes
\end{description}

\subsubsection{UI Appearance Module (\mref{mUI})}

\begin{description}
	\item[Secrets:]The format and appearance of the UI elements.
	\item[Services:]Allows the user to have an uncluttered and pleasant 
	experience with the software.
	\item[Implemented By:] MacSidenotes
\end{description}

\subsection{Software Decision Module}

\subsubsection{Update Master List Module (\mref{mUpdate})}

\begin{description}
	\item[Secrets:]How the Notes are fetched from storage and added to the 
	Master List.
	\item[Services:]Loads all previously saved Notes into the Master List.
	\item[Implemented By:] MacSidenotes
\end{description}

\subsubsection{Save Note Module (\mref{mSave})}

\begin{description}
	\item[Secrets:]The format and location of saved Notes.
	\item[Services:]Allows the user to save notes for future viewing.
	\item[Implemented By:] MacSidenotes
\end{description}

\subsubsection{URL Fetch Module (\mref{mURL})}

\begin{description}
	\item[Secrets:]The technique by which the URL is fetched.
	\item[Services:]Allows each Note to be saved alongside an accompanying URL.
	\item[Implemented By:] MacSidenotes
\end{description}

\subsubsection{Master List Visibility Module (\mref{mVisibility})}

\begin{description}
	\item[Secrets:]The technique by which the visibility of the Master List is 
	determined.
	\item[Services:]Allows the user to toggle the Master List on and off.
	\item[Implemented By:] MacSidenotes
\end{description}

\section{Traceability Matrix} \label{SecTM}

This section shows two traceability matrices: between the modules and the
requirements and between the modules and the anticipated changes.

% the table should use mref, the requirements should be named, use something
% like fref
\begin{table}[H]
\centering
\begin{tabular}{p{0.2\textwidth} p{0.6\textwidth}}
\toprule
\textbf{Req.} & \textbf{Modules}\\
\midrule
F.1 & \mref{mHH}\\
F.2 & \mref{mUI}\\
F.3 & \mref{mSave}, \mref{mTrigger}\\
F.4 & \mref{mUI}, \mref{mVisibility}\\
F.5 & \mref{mVisibility}\\
NF.1 & \mref{mUI}\\
NF.2 & \mref{mUI}\\
NF.3 & \mref{mUI}\\
NF.4 & \mref{mSave}, \mref{mUpdate}, \mref{mVisibility}\\
NF.5 & \mref{mUI}\\
NF.6 & \mref{mHH}, \mref{mUI}, \mref{mSave}, \mref{mUpdate}, 
\mref{mVisibility}\\
NF.7 & \mref{mHH}\\
NF.8 & \mref{mHH}\\
NF.9 & \mref{mHH}\\
NF.10 & \mref{mSave}\\
NF.11 & \mref{mHH}\\
NF.12 & \mref{mUI}\\
NF.13 & \mref{mUI}, \mref{mSave}\\
NF.14 & \mref{mUI}\\
\bottomrule
\end{tabular}
\caption{Trace Between Requirements and Modules}
\label{TblRT}
\end{table}

\begin{table}[H]
\centering
\begin{tabular}{p{0.2\textwidth} p{0.6\textwidth}}
\toprule
\textbf{AC} & \textbf{Modules}\\
\midrule
\acref{acTrigger} & \mref{mTrigger}\\
\acref{acUI} & \mref{mUI}\\
\acref{acSave} & \mref{mSave}\\
\acref{acVisibility} & \mref{mVisibility}\\
\acref{acUpdate} & \mref{mUpdate}\\
\acref{acURL} & \mref{mURL}\\
\acref{acNotify} & \mref{mNotify}\\
\bottomrule
\end{tabular}
\caption{Trace Between Anticipated Changes and Modules}
\label{TblACT}
\end{table}

\section{Use Hierarchy Between Modules} \label{SecUse}

In this section, the uses hierarchy between modules is
provided. \citet{Parnas1978} said of two programs A and B that A {\em uses} B if
correct execution of B may be necessary for A to complete the task described in
its specification. That is, A {\em uses} B if there exist situations in which
the correct functioning of A depends upon the availability of a correct
implementation of B.  Figure \ref{FigUH} illustrates the use relation between
the modules. It can be seen that the graph is a directed acyclic graph
(DAG). Each level of the hierarchy offers a testable and usable subset of the
system, and modules in the higher level of the hierarchy are essentially simpler
because they use modules from the lower levels.

\begin{figure}[H]
\centering
\includegraphics[width=0.7\textwidth]{3XA3UsesRelationship.png}
\caption{Use hierarchy among modules}
\label{FigUH}
\end{figure}

\section*{References}

\bibliographystyle {plainnat}
\bibliography {MG}

\end{document}