\documentclass[12pt, titlepage]{article}

\usepackage{booktabs}
\usepackage{tabularx}
\usepackage{hyperref}
\usepackage{graphicx}
\usepackage{float}
\usepackage{mdframed}
\usepackage{framed}
\usepackage{fancybox}
\newcolumntype{N}{>{\centering\arraybackslash}m{1.5cm}}
\newcolumntype{L}{>{\centering\arraybackslash}m{2.2cm}}
\newcolumntype{C}{>{\centering\arraybackslash}m{4.4cm}}

\hypersetup{
    colorlinks,
    citecolor=black,
    filecolor=black,
    linkcolor=red,
    urlcolor=blue
}
\usepackage[round]{natbib}



\title{SE 3XA3: Requirements Document\\MacSidenotes}

\author{Team 4
		\\ Josh Mitchell mitchjp3
		\\ Matthew Shortt shorttmk
}

\date{\today}

%\input{../Comments}

\begin{document}

\maketitle

\pagenumbering{roman}
\tableofcontents
\listoftables
\listoffigures



\begin{table}[bp]
\caption{\bf Revision History}
\begin{tabularx}{\textwidth}{p{3cm}p{2cm}X}
\toprule {\bf Date} & {\bf Version} & {\bf Notes}\\
\midrule
Oct 6th, 2016 & 0.5 & Section 2 Complete\\
%Date 2 & 1.1 & Notes\\
\bottomrule
\end{tabularx}
\end{table}

\newpage

\pagenumbering{arabic}

%\newmdenv[linecolor=black]{reqbox}

This document describes the requirements for ....  The template for the Software
Requirements Specification (SRS) is a subset of the Volere
template~\citep{RobertsonAndRobertson2012}.  If you make further modifications
to the template, you should explicity state what modifications were made.

\section{Project Drivers}

\subsection{The Purpose of the Project}

\subsection{The Stakeholders}

\subsubsection{The Client}

\subsubsection{The Customers}

\subsubsection{Other Stakeholders}

\subsection{Mandated Constraints}

\subsection{Naming Conventions and Terminology}

\subsection{Relevant Facts and Assumptions}

User characteristics should go under assumptions.

\section{Functional Requirements}

\subsection{The Scope of the Work and the Product}

Currently in the Chrome Web Store, there exists a few extensions that function 
similarly to MacSidenotes. However, each either lacks a feature of this 
product, or suffers from bugs and is poorly reviewed.\\
\\
MacSidenotes' combination of simplicity, dependable functionality and ability 
to link back to web pages from the Note List set it apart from the current 
offerings.

\subsubsection{The Context of the Work}

\begin{figure}[H]
	\centering
	\includegraphics[width=\textwidth]{images/WorkContextDiagram.png}
	\caption{Work Context Diagram}
\end{figure}


\subsubsection{Work Partitioning}
\begin{table}[H]
		\setlength{\extrarowheight}{1ex}
	\caption {\bf Business Event List}
	\begin{tabularx}{\textwidth}{N|L|C|C}
		{\bf Event \#} & {\bf Event Name} & {\bf Input and Output} & {\bf 
		Summary of BUC}\\
		\hline
		1. &User clicks extension icon & Button click event (in) \newline 
		Sidebar view (out) & Extension sidebar appears when icon is clicked.\\
		2. &User types in sidebar & Keystrokes (in) \newline Text in sidebar 
		(out) & Note is displayed as it is being entered.\\
		3. &User clicks save icon & Button click event (in) \newline User's 
		note (in) \newline Webpage URL (in) \newline Save message (out) 
		\newline Note \& URL (out)& Save user's note in local browser storage.\\
		4. &User clicks Note List button & Button click event (in) \newline 
		Note List view (out) & Show list of previous notes.\\
		5. &User clicks note in Note List & Button click event (in) \newline 
		HTTP request (out) \newline Webpage response (in) \newline Webpage 
		view (out) & Send user to URL attached to previously saved note.
	\end{tabularx}

\end{table}

\subsubsection{Individual Product Use Cases}
{\bf 1.} User wishes to create a note for a Wikipedia article (or any webpage) 
they are viewing, so they click the MacSidenotes icon in their chrome browser. 
The notes sidebar appears on the right side of the screen\\
{\bf 2.} User begins to type their note, "Section 7 is relevant for thesis 
paper" and it appears in the sidebar.\\
{\bf 3.} User clicks the save icon, and their note is saved in local browser 
storage along with the URL to that article.\\
{\bf 4.} User wishes to view all previous notes, so they click the MacSidenotes 
icon followed by the Notes List icon. A list of notes the user has recorded in 
the past appear, alongside their respective source URLs.\\
{\bf 5.} User wishes to recall a specific previous note, so they click on that 
note in the Notes List. Chrome then takes them to the URL associated with that 
note, displaying the note in the sidebar as it appeared when they saved it.

\subsection{Functional Requirements}
\begin{framed}
	
	\begin{center}
		
		\begin{tabular}{ l c r }
			Requirement \#: F.1 & Requirement Type: 9 & Event/Use case \#: 1\\
		\end{tabular} \\
	\end{center}
	\textbf{Description:} The product must reveal a typing window "sidebar" on 
	the right hand side of the chrome window when the extension icon is 
	clicked. \\
	\\
	\textbf{Rationale:} Users need a location to type their notes. \\
	\\
	\textbf{Originator:} Josh Mitchell \\
	\\
	\textbf{Fit Criterion:} The window shall appear in the correct location 
	within 1 second of the user clicking on the icon. \\
	
	\begin{tabular}{ll}
		\textbf{Customer Satisfaction:} 5 & \textbf{Customer Dissatisfaction:} 5 \\
		\textbf{Priority:} Very High & \textbf{Conflicts:} None\\
	\end{tabular} \\
	\\
	\textbf{Supporting Materials:} None \\
	\textbf{History:} Created October 6th, 2016
	
\end{framed}
\begin{framed}
	
	\begin{center}
		
		\begin{tabular}{ l c r }
			Requirement \#: F.2 & Requirement Type: 9 & Event/Use case \#: 2\\
		\end{tabular} \\
	\end{center}
	\textbf{Description:} The product must allow the user to type their notes 
	into the sidebar.\\
	\\
	\textbf{Rationale:} Typing is a fast and simple way to record notes. \\
	\\
	\textbf{Originator:} Josh Mitchell \\
	\\
	\textbf{Fit Criterion:} The user's exact keyboard input shall appear in the 
	sidebar within 0.1 seconds of the key being pressed. \\
	
	\begin{tabular}{ll}
		\textbf{Customer Satisfaction:} 5 & \textbf{Customer Dissatisfaction:} 
		5 \\
		\textbf{Priority:} Very High & \textbf{Conflicts:} None\\
	\end{tabular} \\
	\\
	\textbf{Supporting Materials:} None \\
	\textbf{History:} Created October 6th, 2016
	
\end{framed}
\begin{framed}
	
	\begin{center}
		
		\begin{tabular}{ l c r }
			Requirement \#: F.3 & Requirement Type: 9 & Event/Use case \#: 3\\
		\end{tabular} \\
	\end{center}
	\textbf{Description:} Upon the Save Note button being clicked, the product 
	must save the user's note in local browser storage alongside the URL of the 
	web page they were viewing that the time. \\
	\\
	\textbf{Rationale:} Saving notes for future reference greatly increases the 
	usefulness of the extension. \\
	\\
	\textbf{Originator:} Josh Mitchell \\
	\\
	\textbf{Fit Criterion:} After pressing save, Chrome's local storage should 
	append the exact content of the user's note and its associated URL to the 
	list of saved notes. This operation will take under 5 seconds to complete. 
	\\
	
	\begin{tabular}{ll}
		\textbf{Customer Satisfaction:} 4 & \textbf{Customer Dissatisfaction:} 
		4 \\
		\textbf{Priority:} High & \textbf{Conflicts:} None\\
	\end{tabular} \\
	\\
	\textbf{Supporting Materials:} None \\
	\textbf{History:} Created October 6th, 2016
	
\end{framed}
\begin{framed}
	
	\begin{center}
		
		\begin{tabular}{ l c r }
			Requirement \#: F.4 & Requirement Type: 9 & Event/Use case \#: 4\\
		\end{tabular} \\
	\end{center}
	\textbf{Description:} The product must allow the user to see all previously 
	saved notes by clicking the Notes List button.\\
	\\
	\textbf{Rationale:} Viewing previous notes greatly increases the 
	extension's usefulness \\
	\\
	\textbf{Originator:} Josh Mitchell \\
	\\
	\textbf{Fit Criterion:} The list of notes shall appear in the correct 
	location within 1 second of the user clicking on the icon. \\
	
	\begin{tabular}{ll}
		\textbf{Customer Satisfaction:} 4 & \textbf{Customer Dissatisfaction:} 
		4 \\
		\textbf{Priority:} High & \textbf{Conflicts:} None\\
	\end{tabular} \\
	\\
	\textbf{Supporting Materials:} None \\
	\textbf{History:} Created October 6th, 2016
	
\end{framed}
\begin{framed}
	
	\begin{center}
		
		\begin{tabular}{ l c r }
			Requirement \#: F.5 & Requirement Type: 9 & Event/Use case \#: 5\\
		\end{tabular} \\
	\end{center}
	\textbf{Description:} If the user clicks on a previous note, the product 
	must force Chrome to request the webpage associated with that note's URL.\\
	\\
	\textbf{Rationale:} The ability to revisit noted webpages greatly increases 
	the extension's usefulness \\
	\\
	\textbf{Originator:} Josh Mitchell \\
	\\
	\textbf{Fit Criterion:} The HTTP request must be sent within 1 second of 
	the user clicking the note. \\
	
	\begin{tabular}{ll}
		\textbf{Customer Satisfaction:} 3 & \textbf{Customer Dissatisfaction:} 
		3 \\
		\textbf{Priority:} Moderate & \textbf{Conflicts:} None\\
	\end{tabular} \\
	\\
	\textbf{Supporting Materials:} None \\
	\textbf{History:} Created October 6th, 2016
	
\end{framed}

\section{Non-functional Requirements}

\subsection{Look and Feel Requirements}


\begin{framed}

	\begin{center}
		
		\begin{tabular}{ l c r }
			Requirement \#: NF.1 & Requirement Type: 10 & Event/Use case \#: \\
		\end{tabular} \\
	\end{center}
	\textbf{Description:} The product should have an attractive html interface. \\
	\\
	\textbf{Rationale:} The product must be aesthetically pleasing and easy to 
	use to benefit the end-users \\
	\\
	\textbf{Originator:} Matthew Shortt \\
	\\
	\textbf{Fit Criterion:} Stakeholder satisfaction regarding the appearance  \\

	\begin{tabular}{ll}
		\textbf{Customer Satisfaction:} 5 & \textbf{Customer Dissatisfaction:} 5 \\
		\textbf{Priority:} High & \textbf{Conflicts:} None\\
	\end{tabular} \\
	\\
	\textbf{Supporting Materials:} None \\
	\textbf{History:} Created October 5th, 2016

\end{framed}


\subsection{Usability and Humanity Requirements}

\subsection{Performance Requirements}

\subsection{Operational and Environmental Requirements}

\subsection{Maintainability and Support Requirements}

\subsection{Security Requirements}

\subsection{Cultural Requirements}

\subsection{Legal Requirements}

\subsection{Health and Safety Requirements}

This section is not in the original Volere template, but health and safety are
issues that should be considered for every engineering project.

\section{Project Issues}

\subsection{Open Issues}
It is not yet known if our inexperience with JavaScript will be a significant hurdle in the development of this project.

\subsection{Off-the-Shelf Solutions}
\subsubsection{Ready-Made Products}
	\href{https://chrome.google.com/webstore/detail/sticky-notes-just-popped/plpdjbappofmfbgdmhoaabefbobddchk}
	{Sticky Notes - Just popped up!}\\
	\href{https://chrome.google.com/webstore/detail/memo-notepad/nmoihkoninaoanjobiiknmgenhpaecec}
	{Memo Notepad}\\
	\href{https://chrome.google.com/webstore/detail/google-keep-notes-and-lis/hmjkmjkepdijhoojdojkdfohbdgmmhki?hl=en}
	{Google Keep - notes and lists}
	
\subsubsection{Reusable Components}
	\href{https://jquery.com}
	{jQuery}\\
	\href{http://jqueryui.com/}
	{jQuery UI}
	
\subsubsection{Products That Can Be Copied}
	\href{https://github.com/sidenotes/sidenotes}
	{sidenotes}\\
	*Note: this is the existing extension that this project is based on. It is currently in a non-functioning state, although some insight into extension design can be gleaned from it.
\subsection{New Problems}

\subsection{Tasks}

\subsection{Migration to the New Product}

\subsection{Risks}

\subsection{Costs}

\subsection{User Documentation and Training}

\subsection{Waiting Room}

\subsection{Ideas for Solutions}

\bibliographystyle{plainnat}

\bibliography{SRS}

\newpage

\section{Appendix}

This section has been added to the Volere template.  This is where you can place
additional information.

\subsection{Symbolic Parameters}

The definition of the requirements will likely call for SYMBOLIC\_CONSTANTS.
Their values are defined in this section for easy maintenance.


\end{document}