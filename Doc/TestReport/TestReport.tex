\documentclass[12pt, titlepage]{article}

\usepackage{booktabs}
\usepackage{tabularx}
\usepackage{hyperref}
\usepackage[normalem]{ulem}
\usepackage{color}
\usepackage{float}
\hypersetup{
    colorlinks,
    citecolor=black,
    filecolor=black,
    linkcolor=red,
    urlcolor=blue
}
\usepackage[round]{natbib}

\title{SE 3XA3: Test Report\\MacSidenotes}

\author{Team \#04
		\\ Josh Mitchell mitchjp3
		\\ Matthew Shortt shorttmk
}

\date{\today}

%\input{../Comments}

\begin{document}

\maketitle
\pagenumbering{roman}
\tableofcontents
\listoftables
\listoffigures

\begin{table}[h]
\caption{\bf Revision History}
\begin{tabularx}{\textwidth}{p{3cm}p{2cm}X}
\toprule {\bf Date} & {\bf Version} & {\bf Notes}\\
\midrule
Dec 7th & 1.0 & Revision 1\\
\bottomrule
\end{tabularx}
\end{table}

\newpage

\pagenumbering{arabic}

This document describes the implementation of the 
\href{https://gitlab.cas.mcmaster.ca/macsidenotes/macsidenotes/blob/master/Doc/TestPlan/TestPlan.pdf}{MacSidenotes
 Test Plan}. It covers: the testing of Functional and Non-Functional 
 Requirements, System and Unit Testing, and the tracing of Test cases to the 
 modules and requirements.

\section{Functional Requirements Evaluation}
\subsection{Sidebar - Typing Window}

\begin{enumerate}
	
	\item{Sidebar Open\\}
	
	Type: Functional, Dynamic, Manual
	
	Initial State: Default Chrome browser window.
	
	Input: Extension icon click.
	
	Expected Results: Sidebar will appear at the top right of the page.
	
	Actual Results Match Expected Results: Yes
	
	\item{Sidebar Close-1\\}
	
	Type: Functional, Dynamic, Manual
	
	Initial State: Sidebar appears at the top right of the page.
	
	Input: Extension icon click.
	
	Expected Results: Sidebar disappears leaving solely the browser page.
	
	Actual Results Match Expected Results: Yes
	
	\item{Sidebar Close-2\\}
	
	Type: Functional, Dynamic, Manual
	
	Initial State: Sidebar appears at the top right of the page.
	
	Input: Click event on browser page.
	
	Expected Results: Sidebar disappears leaving solely the browser page.
	
	Actual Results Match Expected Results: Yes
	
\end{enumerate}

\subsection{Note Taking}

\paragraph{Typing}

\begin{enumerate}
	
	\item{Input 1\\}
	
	Type: Functional, Dynamic, Manual
	
	Initial State: Blank text area.
	
	Input: User keyboard input.
	
	Expected Results: User keyboard input.
	
	Actual Results Match Expected Results: Yes
	
\end{enumerate}

\subsection{Note Saving}

\paragraph{Save Note}

\begin{enumerate}
	
	\item{Save Note - 1\\}
	
	Type: Functional, Dynamic, Manual
	
	Initial State: Text window with keyboard input.
	
	Input: Click event on 'Save Note' button.
	
	Expected Results: Confirmation Message 'Note Saved!'
	
	Actual Results Match Expected Results: Yes
	
\end{enumerate}

\subsection{Master List}

\paragraph{List Management}

\begin{enumerate}
	
	\item{List View - Open\\}
	
	Type: Functional, Dynamic, Manual
	
	Initial State: Sidebar with solely text area and buttons. 
	
	Input: Click event on 'List' button.
	
	Expected Results: Master List shown below the contents of the Sidebar.
	
	Actual Results Match Expected Results: Yes
	
	\item{List View - Closed\\}
	
	Type: Functional, Dynamic, Manual
	
	Initial State: Sidebar with text area, buttons and Master List. 
	
	Input: Click event on 'List' button.
	
	Expected Results: Master List disappears from the Sidebar.
	
	Actual Results Match Expected Results: Yes
	
\end{enumerate}

\subsection{Revisit URL}

\paragraph{Link to Previous Notes}

\begin{enumerate}
	
	\item{Link to Previous Note\\}
	
	Type: Functional, Dynamic, Manual
	
	Initial State: Master List contains at least 1 note and is visible. 
	
	Input: Click event on a URL in Master List
	
	Expected Results: New Chrome tab opens at that URL.
	
	Actual Results Match Expected Results: Yes
	
\end{enumerate}

\subsection{Note Deletion}

\paragraph{Delete Note}

\begin{enumerate}
	
	\item{Delete Note\\}
	
	Type: Functional, Dynamic, Manual
	
	Initial State: Text Area with a note saved to it. 
	
	Input: Click event on 'Delete Note' button.
	
	Expected Results: Deletion confirmation message. 
	
	Actual Results Match Expected Results: Yes
	
	\item{Delete Confirm - YES\\}
	
	Type: Functional, Dynamic, Manual
	
	Initial State: 'Delete Note' has been clicked prompting confirmation. 
	
	Input: Click event on 'YES' button.
	
	Expected Results: Confirmation of deletion message. 
	
	Actual Results Match Expected Results: Yes
	
	\item{Delete Confirm - NO\\}
	
	Type: Functional, Dynamic, Manual
	
	Initial State: 'Delete Note' has been clicked prompting confirmation. 
	
	Input: Click event on 'NO' button.
	
	Expected Results: Note is not deleted. 
	
	Actual Results Match Expected Results: Yes	
	
\end{enumerate}
\section{Nonfunctional Requirements Evaluation}

\subsection{Look and Feel}

\paragraph{Look}

\begin{enumerate}
	
	\item{Sidebar Size\\}
	
	Type: Manual, Dynamic
	
	Initial State: Default Chrome browser window.
	
	Input: Extension icon click.
	
	Output: Sidebar popup.
	
	Expected Results: Sidebar will appear at the top right of the page and does 
	not take up more than SIDEBAR\_WIDTH.
	
	Actual Results Match Expected Results: Yes	
	
\end{enumerate}

\subsection{Usability and Humanity Requirements}

\paragraph{Usability \& Humanity}

\begin{enumerate}
	
	\item{Communication - English\\}
	
	Type: Manual, Static, Dynamic
	
	Initial State: Varied
	
	Input: N/A
	
	Output: Messages in English
	
	Expected Results: All text is understandable and in English
	
	Actual Results Match Expected Results: Yes 
	
\end{enumerate}

\subsection{Performance}

\paragraph{Speed}

\begin{enumerate}
	
	\item{Extension Response\\}
	
	Type: Manual, Dynamic
	
	Initial State: Varied
	
	Input: User input.
	
	Output: Extension output.
	
	Expected Results: Extension use is not too slow as to interrupt user's flow.
	
	Actual Results Match Expected Results: Yes
	
\end{enumerate}
	
\section{Comparison to Existing Implementation}	

The original project 
\href{https://github.com/sidenotes/sidenotes}{Sidenotes} is defunct and no 
longer works. Sidenotes relied on a Dropbox API that is now deprecated, and 
thus MacSidenotes can't be compared to the original implementation.

\section{Automated Unit Testing}

Due to the nature of the design, all unit tests are automated using Jasmine.



\subsection{clickCounter}

	Type: Unit, Dynamic, Automated

	Initial State: Varied

	Input: Click event on 'Show List' button.

	Expected Results: numClicks incremented

	Actual Results Match Expected Results: Yes
	
\subsection{showList}

	Type: Unit, Dynamic, Automated

	Initial State: Master List not visible

	Input: 2 Click events on 'Show List' button.

	Expected Results: Master List appear, then disappear

	Actual Results Match Expected Results: Yes

\subsection{emptyMasterList}

	Type: Unit, Dynamic, Automated

	Initial State: Master List has no items

	Input: Add items to Master List, then call emptyMasterList()

	Expected Results: Remove all rows from Master List 

	Actual Results Match Expected Results: Yes

\subsection{showNotice}

	Type: Unit, Dynamic, Automated

	Initial State: saveNotice and deleteNotice icons not visible

	Input: 2 Click events on 'Save Note' and 'Delete Note" buttons.

	Expected Results: saveNotice and deleteNotice icons appear

	Actual Results Match Expected Results: Yes

\section{Changes Due to Testing}
		
	The most significant changes to the extension were made after completing 
	Usability Testing. Users suggested that the "http://www." be removed from 
	the URL to increase the aesthetic appeal of the UI.\\
	Users also showed interest in a 'flat' design to the UI buttons and a more 
	cohesive interface. These ideas were incorporated into the design of the 
	extension.
		
\section{Trace to Requirements}	
	\begin{table}[H]
		\begin{center}
		\setlength{\extrarowheight}{1ex}
		\caption {\bf Trace Back to Requirements}
		\begin{tabularx}{\textwidth}{c|c}
			{\bf Test \#} & {\bf Requirement \#}\\
			\hline
			1.1 & F.1 \\
			1.2 & F.2 \\
			1.3 & F.3 \\
			1.4 & F.4 \\
			1.5 & F.5 \\
			1.6 & F.6 \\
			2.1 & NF.2 \\
			2.1 & NF.5 \\
			2.1 & NF.6 \\
		\end{tabularx}
	\end{center}
	\end{table}
		
\section{Trace to Modules}
	
	\begin{table}[H]	
		\setlength{\extrarowheight}{1ex}
		\caption {\bf Trace Back to Modules}
		\begin{tabularx}{\textwidth}{c|c}
			{\bf Test \#} & {\bf Module \#}\\
			\hline
			1.1 & M8 \\
			1.2 & M9 \\
			1.3 & M2 \\
			1.4 & M3 \\
			1.5 & M3 \\
			1.6 & M2 \& M6 \\
			2.1 & M8 \\
			2.1 & M8 \\
			2.1 & M1 \\
			4.1 & M3 \\
			4.2 & M3 \\
			4.3 & M4 \\
			4.4 & M6 \& M8 \\
		\end{tabularx}
		
	\end{table}
\section{Code Coverage Metrics}

\subsection{Function Coverage}
	With the exception of 2 functions that are only used for debugging 
	purposes, every function implemented in the extension is called at least 
	once.
	
\subsection{Statement Coverage}
	Not counting the statements within the 2 aforementioned debugging 
	functions, every statement in the extension is executed.
	
\subsection{Branch Coverage}
	All of the branches that are integral to the core functionality of the 
	extension are covered. The only ones that are not covered are within the 
	aforementioned debugging functions or would not break the extension if they 
	malfunctioned (eg. text truncating).
	
\subsection{Condition Coverage}
	Due to the lack of total Branch Coverage (as addressed above), total 
	Condition Coverage is not achieved. The most important boolean 
	sub-expressions are evaluated to both true and false, though.

\section{Appendix}

\subsection{Symbolic Parameters}

The definition of the test cases will call for SYMBOLIC\_CONSTANTS.
Their values are defined in this section for easy maintenance.

\begin{table}[!htbp]
	\caption{\textbf{Table of Symbolic Parameters}} \label{Table}
	
	\begin{tabularx}{\textwidth}{p{5cm}X}
		\toprule
		\textbf{Symbolic Constants} & \textbf{Value}\\
		\midrule
		\sout{RESPONSE\_TIME} & \sout{The maximum amount of time the system has 
		to respond to a user interaction. This number is 2 seconds.}  \\
		SIDEBAR\_WIDTH & The maximum width that the Sidebar should be upon 
		first opened. The maximum sidebar width is 30\% of a users full browser 
		size.\\
		\bottomrule
	\end{tabularx}
	
\end{table}	

\subsection{Usability Survey Questions}

During October of 2016 Matthew asked his two roommates about the usability of 
the extension. At this point the product was at the proof of concept stage. \\
\\
Some of the questions asked were:\\
\\
What do you like about the product?\\
What do you think could be improved?\\
What features would you like to see added to the product?\\
Would you use the product?\\
\\
Both roommates liked the product as-is due to the facility in which it can be 
used in tandem with it's practicality. Some suggestion put forward included:\\
\\
Bullet Points\\
Font Colours/Sizes/Effects\\
Highlights\\
The ability to add pictures\\
Notification if page has note on it already\\
Titles for notes\\
\\
Overall the meeting was very successful and produced many great ideas that 
could find their way onto the final product. Both roommates said they would 
definitely use the product. 

\bibliographystyle{plainnat}

%\bibliography{SRS}

\end{document}