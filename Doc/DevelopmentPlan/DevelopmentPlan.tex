\documentclass{article}

\usepackage{booktabs}
\usepackage{tabularx}
\usepackage{hyperref}
\usepackage{float}

\title{SE 3XA3: Development Plan\\MacSidenotes}

\author{Team 04
	\\ Josh Mitchell mitchjp3
	\\ Matthew Shortt shorttmk
}

\date{}

%\input{../Comments}

\begin{document}
	
	\begin{table}[hp]
		\caption{Revision History} \label{TblRevisionHistory}
		\begin{tabularx}{\textwidth}{llX}
			\toprule
			\textbf{Date} & \textbf{Developer(s)} & \textbf{Change}\\
			\midrule
			Sept 26, 2016 & Josh, Matt & Layout \& Section 1\\
			Sept 28, 2016 & Matt & Sections 2, 4 \& 6\\
			Sept 29, 2016 & Josh & Sections 3, 5, 7 \& 8\\
			Oct 17, 2016 & Matt & More specific proof of concept\\
			\bottomrule
		\end{tabularx}
	\end{table}
	
	\newpage
	
	\maketitle
	
	MacSidenotes is a Chrome Extension for taking notes on any web site. Your 
	notes are linked with the URL and will appear upon your next visit. Works 
	offline using your browser's local storage.
	
	\section{Team Meeting Plan}
	\subsubsection*{When}
	Monday 6:30-8:30 \\
	Thursday 4:30-6:30
	\subsubsection*{Where}
	Josh's House
	\subsubsection*{Frequency}
	Weekly, after each 3XA3 lab section
	\subsubsection*{Agenda}
	Each meeting will begin by recapping the progress made on homework from the 
	last meeting. If any homework has not been completed, we will discuss the 
	problems or setbacks incurred and how to overcome them. Then homework for 
	the next meeting will be discussed and assigned. Each member will record 
	their 
	homework in a personal agenda.
	
	\section{Team Communication Plan}
	
	Communication will primarily take place through Facebook using their 
	messaging service. This 
	works perfectly for setting up meetings and getting in touch quickly. By exchanging 
	cell phone numbers, contact can also be made via text message if an 
	emergency arises. 
	
	\section{Team Member Roles}
	Josh will assume the role of Team Leader, although Matt and Josh will 
	collaborate on all executive 
	decisions. Given Matt's greater experience with Javascript, he will act as 
	JS expert and primary developer. As Matt and Josh's experience with 
	software 
	documentation, Git and LaTeX are similar, they will take on approximately 
	even roles within these facets of the development process. Each will manage 
	the documentation of code they write. 
	\section{Git Workflow Plan}
	
	Centralized Git workflow plan. This workflow best suits the style 
	and approach to this project as it is the most straightforward method for 
	two individuals 
	who have little personal experience with Git. Tags will be used at project
	milestones (Development Plan, Requirement Document, etc..) to identify the 
	'final'	submission for that certain accomplishment. 
	
	\section{Proof of Concept Demonstration Plan}
	The most consequential risk to the proof of concept demonstration is being 
	unable to write the necessary JavaScript code. Matt and Josh have 
	little and no experience, respectively, writing in JS. They must learn at 
	least the basics of JS syntax before development can begin.\\
	
	Another foreseen difficulty is automating the testing. Given the UI-centric 
	nature of the project, automating the test cases is different than in 
	previous courses.\\
	
	Both of these issues can be overcome by careful education in the early 
	stages of development. Documentation and tutorials for these systems abound 
	online, and only require time to learn.\\
	
	For the proof of concept demo a completed section of the extension will be 
	demonstrated. This section will include a sidebar that will pop-up when the icon is 
	clicked. The note that is written in the sidebar shall be able to be saved locally, along 
	with a reference to the URL that the note was written on. 
	\section{Technology}
	
	HTML, CSS and JavaScript will be used to code the application. 'SublimeText 
	2' is a user 
	friendly
	IDE that will be used for the project. The testing framework will be 
	JsUnit. 
	Documentation will be generated in LaTeX using TexStudio and 
	the Gantt Chart will be produced using GanttProject.
	
	\section{Coding Style}
	
	HTML and CSS for this project will be written using 
	\href{http://www.w3schools.com/html/html5_syntax.asp}
	{w3schools.com HTML5 Style Guide and Coding Conventions.}\\
	
	JS for this project will be written using 
	\href{http://www.w3schools.com/js/js_conventions.asp}
	{w3schools.com JavaScript Style Guide and Coding Conventions} with one 
	exception. Indentation will be achieved using tabs instead of spaces.
	
	\section{Project Schedule}
	 This project's Gantt chart can be found in the Gitlab repo in both .gan 
	 and .pdf formats.
		\begin{table}[H]
			\begin{tabularx}{\textwidth}{cc}		
				\href{https://gitlab.cas.mcmaster.ca/macsidenotes/macsidenotes/blob/master/ProjectSchedule/MacSidenotesProjectSchedule.gan}
				{Gantt Chart .gan} & 
				\href{https://gitlab.cas.mcmaster.ca/macsidenotes/macsidenotes/blob/master/ProjectSchedule/MacSidenotesProjectSchedule.pdf}
				{Gantt Chart .pdf}
			\end{tabularx}
		\end{table}
	
	\section{Project Review}
	
\end{document}