\documentclass[12pt, titlepage]{article}

\usepackage{booktabs}
\usepackage{tabularx}
\usepackage{hyperref}
\hypersetup{
    colorlinks,
    citecolor=black,
    filecolor=black,
    linkcolor=red,
    urlcolor=blue
}
\usepackage[round]{natbib}

\title{SE 3XA3: Test Plan\\MacSidenotes}

\author{Team 4
		\\ Josh Mitchell mitchjp3
		\\ Matthew Shortt shorttmk
}

\date{\today}

%\input{../Comments}

\begin{document}

\maketitle

\pagenumbering{roman}
\tableofcontents
\listoftables
\listoffigures

\begin{table}[bp]
\caption{\bf Revision History}
\begin{tabularx}{\textwidth}{p{3cm}p{2cm}X}
\toprule {\bf Date} & {\bf Version} & {\bf Notes}\\
\midrule
Oct 30th, 2016 & 0.5 & Sections 2, 4, 5, 6 Added\\
%Date 2 & 1.1 & Notes\\
\bottomrule
\end{tabularx}
\end{table}

\newpage

\pagenumbering{arabic}

This document specifies the tools and techniques that will be used to test the 
adherence of MacSidenotes to its Functional and Non-Functional Requirements. 

\section{General Information}

\subsection{Purpose}

\subsection{Scope}

\subsection{Acronyms, Abbreviations, and Symbols}
	
\begin{table}[hbp]
\caption{\textbf{Table of Abbreviations}} \label{Table}

\begin{tabularx}{\textwidth}{p{3cm}X}
\toprule
\textbf{Abbreviation} & \textbf{Definition} \\
\midrule
Abbreviation1 & Definition1\\
Abbreviation2 & Definition2\\
\bottomrule
\end{tabularx}

\end{table}

\begin{table}[!htbp]
\caption{\textbf{Table of Definitions}} \label{Table}

\begin{tabularx}{\textwidth}{p{3cm}X}
\toprule
\textbf{Term} & \textbf{Definition}\\
\midrule
Note & Text created by the user. Usually intended to be saved and viewed at a  
later time.\\
Sidebar & The popup window that appears when a user clicks on the MacSidenotes 
icon. In the sidebar: users write, save and delete notes, as well as view the 
master list of notes.\\
Master List & The list containing all previously saved notes. It can be viewed 
by clicking the List button.\\
\bottomrule
\end{tabularx}

\end{table}	

\subsection{Overview of Document}

\section{Plan}
	
\subsection{Software Description}

MacSidenotes is a Chrome extension that allows users to create notes for a 
webpage and save them alongside that page's URL. They can then view all 
previous notes in a list and navigate to a URL to continue their previous work.

\subsection{Test Team}

The following project members will be responsible for writing and executing 
tests on MacSidenotes:

\begin{itemize}
	\item Josh Mitchell
	\item Matthew Shortt
\end{itemize}

\subsection{Automated Testing Approach}

Given the UI-driven nature of this project, much of the automated testing 
will be focused on user interaction, which can be simulated with our Testing 
Tool QUnit.
\subsubsection{Whitebox Testing}
Values held in the program will be examined before and after a 
simulated click or keyboard event, to ensure proper updating of system 
variables. These tests are Whitebox or Structural tests, as they require 
knowledge of variable names and the "under-the-hood" storage architecture used 
by the program.
\subsubsection{Blackbox Testing}
The very basic functionality of MacSidenotes involves the appearance and 
disappearance of UI elements. Testing this functionality can be done without 
knowing the internal structure of the program, just the knowledge that the 
"Show List" button should show the user a list.

\subsection{Testing Tools}

QUnit will be the dominant tool for testing the JavaScript functionality of the 
extension. It facilitates automated unit testing, assertions, synchronous and 
asynchronous callbacks as well as testing user actions through simulated mouse 
clicks and keyboard input.

\subsection{Testing Schedule}
		
See Gantt Chart at the following links:

\begin{itemize}
	\item 
	\href{https://gitlab.cas.mcmaster.ca/macsidenotes/macsidenotes/blob/master/ProjectSchedule/MacSidenotesProjectSchedule.gan}{.gan
	 format}
	\item 
	\href{https://gitlab.cas.mcmaster.ca/macsidenotes/macsidenotes/blob/master/ProjectSchedule/MacSidenotesProjectSchedule.pdf}{.pdf
	 format}
\end{itemize}


\section{System Test Description}
	
\subsection{Tests for Functional Requirements}

\subsubsection{Area of Testing1}
		
\paragraph{Title for Test}

\begin{enumerate}

\item{test-id1\\}

Type: Functional, Dynamic, Manual, Static etc.
					
Initial State: 
					
Input: 
					
Output: 
					
How test will be performed: 
					
\item{test-id2\\}

Type: Functional, Dynamic, Manual, Static etc.
					
Initial State: 
					
Input: 
					
Output: 
					
How test will be performed: 

\end{enumerate}

\subsubsection{Area of Testing2}

...

\subsection{Tests for Nonfunctional Requirements}

\subsubsection{Area of Testing1}
		
\paragraph{Title for Test}

\begin{enumerate}

\item{test-id1\\}

Type: 
					
Initial State: 
					
Input/Condition: 
					
Output/Result: 
					
How test will be performed: 
					
\item{test-id2\\}

Type: Functional, Dynamic, Manual, Static etc.
					
Initial State: 
					
Input: 
					
Output: 
					
How test will be performed: 

\end{enumerate}

\subsubsection{Area of Testing2}

...

\section{Tests for Proof of Concept}

\subsection{Area of Testing1}
		
\paragraph{Title for Test}

\begin{enumerate}

\item{test-id1\\}

Type: Functional, Dynamic, Manual, Static etc.
					
Initial State: 
					
Input: 
					
Output: 
					
How test will be performed: 
					
\item{test-id2\\}

Type: Functional, Dynamic, Manual, Static etc.
					
Initial State: 
					
Input: 
					
Output: 
					
How test will be performed: 

\end{enumerate}

\subsection{Area of Testing2}

...

	
\section{Comparison to Existing Implementation}	
	The project that MacSidenotes is derived from 
	(\href{https://github.com/sidenotes/sidenotes}{sidenotes}) is currently 
	unable to be tested. It relies on Dropbox's Datastore API which was retired 
	2 years ago, so it has been defunct since then.\\
	That said, the basic superficial functionality of MacSidenotes can still be 
	contrasted against sidenotes, as the original repository contains a 
	system-level description of its abilities and a 
	\href{https://github.com/sidenotes/sidenotes/blob/master/images/Sidenotes-Screenflow.gif}
	{gif} displaying a user's interaction with the extension.
\section{Unit Testing Plan}
		
\subsection{Unit testing of internal functions}

Each of the below tests requires its own driver. None require a stub.

\subsubsection{deleteNote}
	deleteNote removes the note associated with the URL the user is currently 
	viewing from the master list of notes.\\
	\textbf{To test:} Simulate a click of the Delete Note button with QUnit, 
	then search through Chrome's local storage to ensure no notes exist with a 
	URL that matches the URL the "user" was viewing when deleting that note. If 
	there is no match, the test is successful.
\subsubsection{deleteEmpty}
	deleteEmpty removes all empty notes from the master list.\\
	\textbf{To test:} Call deleteEmpty, search through local storage to ensure 
	no empty notes exist. If no empty notes are found, the test is successful
\subsubsection{clickCounter}
	clickCounter increments numClicks, which counts the number of times the 
	List button has been clicked.\\
	\textbf{To test:} Check the value of numClicks, simulate a click of the 
	List button, 
	check to see if numClicks has incremented by 1. If so, the test is 
	successful.
\subsubsection{showList}
	showList displays the master list of notes to the user. It also removes it 
	from the screen if it is already present.\\
	\textbf{To test:} Simulate a click of the List button, check the visibility 
	of the 
	List element. Simulate another click and check the visibility again. If 
	they do not match, the test is successful.
\subsubsection{updateMasterList}
	updateMasterList appends the note the user has saved to the master list so 
	it can be viewed when the List button is clicked.\\
	\textbf{To test:} Simulate writing a note and clicking the Save Note 
	button. 
	Simulate a click of the List button and check to see if an element of that 
	list exists with the same content and associated URL of the typed note. If 
	both the content and the URL match, the test is successful.
\subsubsection{updateNote}
	If the user has previously saved a note for the URL they are currently 
	viewing, updateNote will display it when the extension icon is clicked.\\
	\textbf{To test:} Simulate the writing and saving of a note. Close 
	MacSidenotes and 
	click on the icon again. Check the value of the sidebar's text area against 
	the typed note. If the content of the text area matches the typed note, the 
	test is successful.
\subsubsection{saveNote}
	saveNote saves the current note in local storage along with it's associated 
	URL.\\
	\textbf{To test:} Simulate the writing and saving of a note. Check local 
	storage to ensure that the master list includes a note that contains the 
	same content and associated URL as the typed note. If both the content and 
	the URL match, the test is successful.
\subsubsection{getURL}
	getURL returns the URL of the webpage the user is currently viewing.\\
	\textbf{To test:} Call getURL and check it's value against Chrome's 
	official code for grabbing URLs found 
	\href{https://developer.chrome.com/extensions/tabs}{here} 

\subsection{Unit testing of output files}		

MacSidenotes does not create output files, as everything is contained within 
the Chrome browser's local storage.

\bibliographystyle{plainnat}

\bibliography{SRS}

\newpage

\section{Appendix}

This is where you can place additional information.

\subsection{Symbolic Parameters}

The definition of the test cases will call for SYMBOLIC\_CONSTANTS.
Their values are defined in this section for easy maintenance.

\subsection{Usability Survey Questions?}

This is a section that would be appropriate for some teams.

\end{document}