\documentclass[12pt, titlepage]{article}

\usepackage{booktabs}
\usepackage{tabularx}
\usepackage{hyperref}
\usepackage[normalem]{ulem}
\usepackage{color}
\hypersetup{
    colorlinks,
    citecolor=black,
    filecolor=black,
    linkcolor=red,
    urlcolor=blue
}
\usepackage[round]{natbib}

\title{SE 3XA3: Test Plan\\MacSidenotes}

\author{Team 4
		\\ Josh Mitchell mitchjp3
		\\ Matthew Shortt shorttmk
}

\date{\today}

%\input{../Comments}

\begin{document}

\maketitle

\pagenumbering{roman}
\tableofcontents
\listoftables
\listoffigures

\begin{table}[bp]
\caption{\bf Revision History}
\begin{tabularx}{\textwidth}{p{3cm}p{2cm}X}
\toprule {\bf Date} & {\bf Version} & {\bf Notes}\\
\midrule
Oct 30th, 2016 & 0.5 & Sections 2, 4, 5, 6 Added\\
\textcolor{blue}{Dec 7th} & \textcolor{blue}{1.0} & \textcolor{blue}{Revision 1 
Modifications}\\
\bottomrule
\end{tabularx}
\end{table}

\newpage

\pagenumbering{arabic}

This document specifies the tools and techniques that will be used to test the 
adherence of MacSidenotes to its Functional and Non-Functional Requirements. 

\section{General Information}

\subsection{Purpose}

The purpose of this document is to create a plan that testers can easily follow in 
order to catch and resolve as many program errors as possible. This plan includes a wide 
variety of test cases that cover both functional and non-functional requirements together 
with automated and manual testing strategies. 

\subsection{Scope}

The Chrome extension that is created for this project is one that is very simple to use and 
very useful for many people.  Since the project is very compact, the scope for our test 
plan will cover the front end and the back end of our implementation. Our front end being 
the GUI the user interacts with, and the back end being what the code does behind the 
scenes, such as using Chrome's localStorage to save users Notes. The scope of the test 
plan will inevitably expand proportionally to the growth of the project, which may include 
an option of saving the Note to the users Google Drive.

\subsection{Acronyms, Abbreviations, and Symbols}
	
\begin{table}[hbp]
\caption{\textbf{Table of Abbreviations}} \label{Table}

\begin{tabularx}{\textwidth}{p{3cm}X}
\toprule
\textbf{Abbreviation} & \textbf{Definition} \\
\midrule
localStorage & Google Chrome's method for storing data locally.\\
JS & JavaScript\\
URL & Uniform Resource Locator\\
HTML & HyperText Markup Language\\
CSS & Cascading Style Sheets\\
POC & Proof Of Concept\\
\bottomrule
\end{tabularx}

\end{table}

\begin{table}[!htbp]
\caption{\textbf{Table of Definitions}} \label{Table}

\begin{tabularx}{\textwidth}{p{3cm}X}
\toprule
\textbf{Term} & \textbf{Definition}\\
\midrule
Note & Text created by the user. Usually intended to be saved and viewed at a  
later time.\\
Sidebar & The popup window that appears when a user clicks on the MacSidenotes 
icon. In the sidebar: users write, save and delete notes, as well as view the 
master list of notes.\\
Master List & The list containing all previously saved notes. It can be viewed 
by clicking the List button.\\
\bottomrule
\end{tabularx}

\end{table}	

\subsection{Overview of Document}

This document is a summation of all the tests that will be performed on our projext to 
ensure that we detect as many errors as possible. This includes tests which are 
automated as well as manual.

\section{Plan}
	
\subsection{Software Description}

MacSidenotes is a Chrome extension that allows users to create notes for a 
webpage and save them alongside that page's URL. They can then view all 
previous notes in a list and navigate to a URL to continue their previous work.

\subsection{Test Team}

The following project members will be responsible for writing and executing 
tests on MacSidenotes:

\begin{itemize}
	\item Josh Mitchell
	\item Matthew Shortt
\end{itemize}

\subsection{Automated Testing Approach}

Given the UI-driven nature of this project, much of the automated testing 
will be focused on user interaction, which can be simulated with our Testing 
Tool \sout{QUnit}\textcolor{blue}{ Jasmine}.
\subsubsection{Whitebox Testing}
Values held in the program will be examined before and after \sout{a 
simulated click or keyboard event} \textcolor{blue}{internal functions are 
called}, to ensure proper updating of system 
variables. These tests are Whitebox or Structural tests, as they require 
knowledge of variable names and the "under-the-hood" storage architecture used 
by the program.
\subsubsection{Blackbox Testing}
The very basic functionality of MacSidenotes involves the appearance and 
disappearance of UI elements. Testing this functionality can be done without 
knowing the internal structure of the program, just the knowledge that the 
"Show List" button should show the user a list.

\subsection{Testing Tools}

\sout{QUnit} \textcolor{blue}{Jasmine} will be the dominant tool for testing 
the JavaScript functionality of the 
extension. It facilitates automated unit testing, assertions, and synchronous 
and asynchronous callbacks \sout{ as well as testing user actions through 
simulated mouse clicks and keyboard input}.

\subsection{Testing Schedule}
		
See Gantt Chart at the following links:

\begin{itemize}
	\item 
	\href{https://gitlab.cas.mcmaster.ca/macsidenotes/macsidenotes/blob/master/ProjectSchedule/MacSidenotesProjectSchedule.gan}{.gan
	 format}
	\item 
	\href{https://gitlab.cas.mcmaster.ca/macsidenotes/macsidenotes/blob/master/ProjectSchedule/MacSidenotesProjectSchedule.pdf}{.pdf
	 format}
\end{itemize}


\section{System Test Description}
	
\subsection{Tests for Functional Requirements}

\subsubsection{Sidebar - Typing Window}
		
\paragraph{Title for Test}

\begin{enumerate}

\item{Sidebar Open\\}

Type: Functional, Dynamic, Manual
					
Initial State: Default Chrome browser window.
					
Input: Extension icon click.
					
Output: Sidebar will appear at the top right of the page.
					
How test will be performed: Test will be performed by clicking the icon for the 
MacSidenotes extension at the top right hand corner of the Google Chrome browser.
					
\item{Sidebar Close-1\\}

Type: Functional, Dynamic, Manual
					
Initial State: Sidebar appears at the top right of the page.
					
Input: Extension icon click.
					
Output: Sidebar disappears leaving solely the browser page.
					
How test will be performed: Test will be performed by clicking the icon for the 
MacSidenotes extension at the top right hand corner of the Google Chrome browser.

\item{Sidebar Close-2\\}

Type: Functional, Dynamic, Manual

Initial State: Sidebar appears at the top right of the page.

Input: Click event on browser page.

Output: Sidebar disappears leaving solely the browser page.

How test will be performed: Test will be performed by clicking anywhere on the 
browser page excluding the extension icon or within the text window.

\end{enumerate}

\subsubsection{Note Taking}

\paragraph{Typing}

\begin{enumerate}
	
	\item{Input 1\\}
	
	Type: Functional, Dynamic, Manual
	
	Initial State: Blank text area.
	
	Input: User keyboard input.
	
	Output: User keyboard input.
	
	How test will be performed: Test will be performed by typing any keyboard input into 
	the text area and verify that the output matches exactly what was typed.
	
\end{enumerate}

\subsubsection{Note Saving}

\paragraph{Save Note}

\begin{enumerate}
	
	\item{Save Note - 1\\}
	
	Type: Functional, Dynamic, Manual
	
	Initial State: Text window with keyboard input.
	
	Input: Click event on 'Save Note' button.
	
	Output: Confirmation Message 'Note Saved!'
	
	How test will be performed: Test will be performed by clicking the 'Save Note' button 
	and verifying that when the note is closed and re-opened it is the same as when the 
	save button	was last clicked. 
	
\end{enumerate}

\subsubsection{Master List}

\paragraph{List Management}

\begin{enumerate}
	
	\item{List View - Open\\}
	
	Type: Functional, Dynamic, Manual
	
	Initial State: Sidebar with solely text area and buttons. 
	
	Input: Click event on 'List' button.
	
	Output: Master List shown below the contents of the Sidebar.
	
	How test will be performed: Test will be performed by clicking the 'List' button located 
	on the extension window and verifying that the Master List appears on the Sidebar. 
	
	\item{List View - Closed\\}
	
	Type: Functional, Dynamic, Manual
	
	Initial State: Sidebar with text area, buttons and Master List. 
	
	Input: Click event on 'List' button.
	
	Output: Master List disappears from the Sidebar.
	
	How test will be performed: Test will be performed by clicking the 'List' button located 
	on the extension window and verifying that the Master List disappears on the Sidebar.  
	
\end{enumerate}

\subsubsection{Revisit URL}

\paragraph{Link to Previous Notes}

\begin{enumerate}
	
	\item{Link to Previous Note\\}
	
	Type: Functional, Dynamic, Manual
	
	Initial State: Master List contains at least 1 note and is visible. 
	
	Input: Click event on a URL in Master List
	
	Output: New Chrome tab opens at that URL.
	
	How test will be performed: Test will be performed by clicking the URL 
	located in the Master List and verifying that Chrome opens a new tab at 
	that URL.
	
\end{enumerate}

\subsubsection{Note Deletion}

\paragraph{Delete Note}

\begin{enumerate}
	
	\item{Delete Note\\}
	
	Type: Functional, Dynamic, Manual
	
	Initial State: Text Area with a note saved to it. 
	
	Input: Click event on 'Delete Note' button.
	
	Output: Deletion confirmation message. 
	
	How test will be performed: Test will be performed by clicking the 'Delete Note' button 
	and assuring that a confirmation message appears saying 'Are you sure you want to 
	delete the note?', along with two choices, 'YES' or 'NO'.
	
	\item{Delete Confirm - YES\\}
	
	Type: Functional, Dynamic, Manual
	
	Initial State: 'Delete Note' has been clicked prompting confirmation. 

	Input: Click event on 'YES' button.
	
	Output: Confirmation of deletion message. 
	
	How test will be performed: Test will be performed by clicking the 'YES' button 
	and assuring that a confirmation message appears saying 'Note Deleted' as well as 
	affirming the note has been removed from Master List. 

	\item{Delete Confirm - NO\\}
	
	Type: Functional, Dynamic, Manual
	
	Initial State: 'Delete Note' has been clicked prompting confirmation. 
	
	Input: Click event on 'NO' button.
	
	Output: Note is not deleted. 
	
	How test will be performed: Test will be performed by clicking the 'NO' button 
	and assuring that a confirmation message appears saying 'Note Was Not Deleted' as 
	well as affirming the note remains in Master List. 	
	
\end{enumerate}

\subsection{Tests for Nonfunctional Requirements}

\subsubsection{Look and Feel}
		
\paragraph{Look}

\begin{enumerate}

\item{Sidebar Size\\}

Type: Manual, Dynamic

Initial State: Default Chrome browser window.

Input: Extension icon click.

Output: Sidebar will appear at the top right of the page.

How test will be performed: Test will be performed by clicking the icon for the 
MacSidenotes extension at the top right hand corner of the Google Chrome browser and 
ensuring that the Sidebar does not take up more than SIDEBAR\_WIDTH		
 
\end{enumerate}

\subsubsection{Usability and Humanity Requirements}

\paragraph{Usability \& Humanity}

\begin{enumerate}

\item{Communication - English\\}

Type: Manual, Static, Dynamic

Initial State: Varied

Input: N/A

Output: N/A

How test will be performed: The test will be performed by activating every message to the 
user and ensuring that they are easily understandable and in English. 

\end{enumerate}

\subsubsection{Performance}

\paragraph{Speed}

\begin{enumerate}
	
	\item{Extension Response\\}
	
	Type: Manual, Dynamic
	
	Initial State: Varied
	
	Input: User input.
	
	Output: Extension output.
	
	How test will be performed: This test will be performed by ensuring that any interaction 
	the user makes with the extension \sout{takes less than RESPONSE\_TIME} 
	\textcolor{blue}{is not too slow as to disrupt their usual flow.}
	
\end{enumerate}

\section{Tests for Proof of Concept}

\subsection{Sidebar}
		
\paragraph{Sidebar Opening}

\begin{enumerate}

\item{Sidebar Opening\\}

Type: Functional, Dynamic, Manual
					
Initial State: Default Chrome browser
					
Input: Click event to the extension icon.
					
Output: Sidebar pops-up.
					
How test will be performed: This will be tested by clicking the extension icon and ensuring 
that the sidebar opens. 

\end{enumerate}

\subsection{Save Feature}

\paragraph{Local Save}

\begin{enumerate}
	
	\item{Saving Note\\}
	
	Type: Functional, Dynamic, Manual
	
	Initial State: Opened Sidebar.
	
	Input: Click event on the 'Save' button.
	
	Output: Confirmation message appears ensuring that note has been save.
	
	How test will be performed: This will be tested by clicking the 'Save' button and 
	ensuring that in the JS  of the project the note and the URL associated with that note 
	is accessible in the localStorage of that user. 
	
\end{enumerate}
	
\section{Comparison to Existing Implementation}	
	The project that MacSidenotes is derived from 
	(\href{https://github.com/sidenotes/sidenotes}{sidenotes}) is currently 
	unable to be tested. It relies on Dropbox's Datastore API which was retired 
	2 years ago, so it has been defunct since then.\\
	That said, the basic superficial functionality of MacSidenotes can still be 
	contrasted against sidenotes, as the original repository contains a 
	system-level description of its abilities and a 
	\href{https://github.com/sidenotes/sidenotes/blob/master/images/Sidenotes-Screenflow.gif}
	{gif} displaying a user's interaction with the extension.
\section{Automated Unit Testing Plan}
		
\subsection{Unit testing of internal functions}

Each of the below tests requires its own driver. None require a stub.
\textcolor{blue}{Some previously planned automated unit test cases had to be 
removed, as the testing framework does not simulate the calls to the Chrome 
extension API. More unit tests that Jasmine can execute were added to 
compensate.}

\subsubsection{deleteNote}
	\sout{deleteNote removes the note associated with the URL the user is 
	currently 
	viewing from the master list of notes.\\
	\textbf{To test:} Simulate a click of the Delete Note button with QUnit, 
	then search through Chrome's local storage to ensure no notes exist with a 
	URL that matches the URL the "user" was viewing when deleting that note. If 
	there is no match, the test is successful.}
\subsubsection{deleteEmpty}
	\sout{deleteEmpty removes all empty notes from the master list.\\
	\textbf{To test:} Call deleteEmpty, search through local storage to ensure 
	no empty notes exist. If no empty notes are found, the test is successful}
\subsubsection{clickCounter}
	clickCounter increments numClicks, which counts the number of times the 
	List button has been clicked.\\
	\textbf{To test:} Check the value of numClicks, \sout{simulate a click of 
	the 
	List button, }\textcolor{blue}{callClickCounter() and}
	check to see if numClicks has incremented by 1. If so, the test is 
	successful.
\subsubsection{showList}
	showList displays the master list of notes to the user. It also removes it 
	from the screen if it is already present.\\
	\textbf{To test:} \sout{Simulate a click of the List 
	button}\textcolor{blue}{Call showList() and clickCounter()}, check the 
	visibility 
	of the 
	List element. \sout{Simulate another click}\textcolor{blue}{Call the 
	functions again} and check the visibility again. If 
	\sout{they do not match} \textcolor{blue}{after the first click it is 
	visible and after the second click it is not}, the test is successful.
\subsubsection{updateMasterList}
	\sout{updateMasterList appends the note the user has saved to the master 
	list so 
	it can be viewed when the List button is clicked.\\
	\textbf{To test:} Simulate writing a note and clicking the Save Note 
	button. 
	Simulate a click of the List button and check to see if an element of that 
	list exists with the same content and associated URL of the typed note. If 
	both the content and the URL match, the test is successful.}
\subsubsection{updateNote}
	\sout{If the user has previously saved a note for the URL they are 
	currently 
	viewing, updateNote will display it when the extension icon is clicked.\\
	\textbf{To test:} Simulate the writing and saving of a note. Close 
	MacSidenotes and 
	click on the icon again. Check the value of the sidebar's text area against 
	the typed note. If the content of the text area matches the typed note, the 
	test is successful.}
\subsubsection{saveNote}
	\sout{saveNote saves the current note in local storage along with it's 
	associated 
	URL.\\
	\textbf{To test:} Simulate the writing and saving of a note. Check local 
	storage to ensure that the master list includes a note that contains the 
	same content and associated URL as the typed note. If both the content and 
	the URL match, the test is successful.}
\subsubsection{getURL}
	\sout{getURL returns the URL of the webpage the user is currently viewing.\\
	\textbf{To test:} Call getURL and check it's value against Chrome's 
	official code for grabbing URLs found 
	\href{https://developer.chrome.com/extensions/tabs}{here} }
\subsubsection{emptyMasterList}
	\textcolor{blue}{emptyMasterList empties the master list of notes.\\
	\textbf{To test:} Add a number of dummy rows to the Master List. Confirm 
	that the number of rows added is as intended. Then call emptyMasterList() 
	and check to see if the List has no rows in it. If it has no rows, the test 
	is successful.}
\subsubsection{showNotice}
\textcolor{blue}{showNotice reveals an icon to the user, notifying them when a 
Note has been saved or deleted.\\
	\textbf{To test:} Check if the saveNotice and deleteNotice icons are not 
	displayed. Call showNotice(), passing both saveNotice and deleteNotice. 
	Check if they are both now displayed. If they were only displayed after the 
	function call, the test is successful.}
\subsection{Unit testing of output files}		

MacSidenotes does not create output files, as everything is contained within 
the Chrome browser's local storage.

\bibliographystyle{plainnat}

\bibliography{SRS}

\newpage

\section{Appendix}

This is where you can place additional information.

\subsection{Symbolic Parameters}

The definition of the test cases will call for SYMBOLIC\_CONSTANTS.
Their values are defined in this section for easy maintenance.

\begin{table}[!htbp]
	\caption{\textbf{Table of Symbolic Parameters}} \label{Table}
	
	\begin{tabularx}{\textwidth}{p{5cm}X}
		\toprule
		\textbf{Symbolic Constants} & \textbf{Value}\\
		\midrule
		RESPONSE\_TIME & The maximum amount of time the system has to respond to a 
		user interaction. This number is 2 seconds.  \\
		SIDEBAR\_WIDTH & The maximum width that the Sidebar should be upon first 
		opened. The maximum sidebar width is 30\% of a users full browser size.\\
		\bottomrule
	\end{tabularx}
	
\end{table}	

\subsection{Usability Survey Questions?}

During October of 2016 Matthew asked his two roommates about the usability of the 
product. At this point the product was at the proof of concept stage. \\
\\
Some of the questions asked were:\\
\\
What do you like about the product?\\
What do you think could be improved?\\
What features would you like to see added to the product?\\
Would you use the product?\\
\\
Both roommates liked the product as-is due to the facility in which it can be used in 
tandem with it's practicality. Some suggestion put forward included:\\
\\
Bullet Points\\
Font Colours/Sizes/Effects\\
Highlights\\
The ability to add pictures\\
Notification if page has note on it already\\
Titles for notes\\
\\
Overall the meeting was very successful and produced many great ideas that could find 
their way onto the final product. Both roommates said they would definitely use the 
product. 


\end{document}