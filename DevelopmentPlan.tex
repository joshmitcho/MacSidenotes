\documentclass{article}

\usepackage{booktabs}
\usepackage{tabularx}

\title{SE 3XA3: Development Plan\\Macsidenotes}

\author{Team 04
	\\ Josh Mitchell mitchjp3
	\\ Matthew Shortt shorttmk
}

\date{}

%\input{../Comments}

\begin{document}
	
	\begin{table}[hp]
		\caption{Revision History} \label{TblRevisionHistory}
		\begin{tabularx}{\textwidth}{llX}
			\toprule
			\textbf{Date} & \textbf{Developer(s)} & \textbf{Change}\\
			\midrule
			Sept 26, 2016 & Josh, Matt & Initial Draft\\
		%	Date2 & Name(s) & Description of changes\\
		%	... & ... & ...\\
			\bottomrule
		\end{tabularx}
	\end{table}
	
	\newpage
	
	\maketitle
	
	Macsidenotes is a Chrome extension that allows users to create notes
	
	\section{Team Meeting Plan}
	\subsubsection*{When}
	Monday 6:30-8:30 \\
	Thursday 4:30-6:30
	\subsubsection*{Where}
	Josh's House
	\subsubsection*{Frequency}
	Weekly, after each 3XA3 lab section
	\subsubsection*{Agenda}
	Each meeting will begin by recapping the progress made on homework from the 
	last meeting. If any homework has not been completed, we will discuss the 
	problems or setbacks incurred and how to overcome them. Then homework for 
	next meeting will be discussed and assigned. Each member will record their 
	homework in a personal agenda.
	
	\section{Team Communication Plan}
	
	We will primarily communicate through Facebook using their messaging service. This 
	works perfectly for setting up meetings and getting in touch quickly. By exchanging 
	contact information we can also get in tough via text if an emergency arises. 
	
	\section{Team Member Roles}
	Given the small size (2) of the development team, there is little need for 
	a single Team Leader; Matt and Josh will collaborate on all executive 
	decisions. Given their relatively equal experience with software 
	documentation, git, LaTeX and Javascript, they will take on approximately 
	even roles within each facet of the development process. Each will manage 
	the documentation of code they write. 
	\section{Git Workflow Plan}
	
	We will be using the Centralized Git workflow plan. This workflow best suited our style 
	and approach to this project as the most straightforward method for two individuals 
	who have little personal experience with Git. Tags will be used at 
	milestones of our 
	project (Develeopment Plan, Requirement Document, etc..) to identify our 'final' 
	submission for that certain accomplishment. 
	
	\section{Proof of Concept Demonstration Plan}
	The most consequential risk to the proof of concept demonstration is being 
	unable to write the necessary Javascript code. Both Matt and Josh have 
	little to no experience writing in JS, so before development can begin, 
	they must learn at least the basics of JS syntax.\\
	Another forseen difficulty for the POCD is 
	\section{Technology}
	
	We will be using JavaScript to code our application. 'Eclipse' and 
	'SublimeText 2' are two 
	IDEs that could be used for our project. We will be using JUnit testing as our testing 
	framework. Finally our documentation will be generated in LaTeX.
	
	\section{Coding Style}
	
	\section{Project Schedule}
	
	Provide a pointer to your Gantt Chart.
	
	\section{Project Review}
	
\end{document}